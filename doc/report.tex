\documentclass[12pt,letterpaper]{article}

\usepackage{amsmath, amsthm, amsfonts, amssymb}
\usepackage{microtype, parskip, graphicx}
\usepackage{multirow, morefloats, rotating}
\usepackage{caption, subcaption}
\usepackage{lineno}

\title{Report on Fossil/Not Fossil}


\begin{document}
\maketitle

\section{Set-Up}

We want to predict the expected diversity of a Macrostrat geological unit. We are focused on the diversity of individual biological classes. Geological units are described in terms of their lithological descriptions, areal extent, location, ``connectedness'' (units above/below), and other values. Lithological description is a compositional variable and thus has annoying properties which can prove difficult to interpret. 

The dataset is split into two sets: the late Ordovician (460.4--445.6) and the Hirnantian (445.6--443.8); the former is the training dataset while the latter is the testing dataset.


\section{Model}

The basics of the models used in this analysis is a hurdle model which is a mixture of a Bernoulli distribution and either a Poisson or Negative-Binomial distribution. The Bernoulli aspect describes the probability of observing 0 species in a geological unit, while the Poisson or Negative-Binomial describes the expected number of species present in that geological unit if there are more than 0.

Both parts of the mixture are modeled as regressions with all of the geological unit covariates as predictors.

The models are fit to the training dataset and the results of which are used to predict the diversity of the Hirnantian geological units. The approximate expected out-of-sample predictions were evaluated through 5 rounds of 5-fold cross-validation.

The models are fit in a fully Bayesian context using the Stan probabilistic programming language. Model fit for the complete training dataset is evaluated through a series of posterior predictive checks.


\section{Results}

\subsection{Full data posterior predictive checks}
\begin{figure}[ht]
  \centering
  \includegraphics[width=\textwidth,height=0.4\textheight,keepaspectratio=true]{figure/}
  \caption{}
  \label{fig:}
\end{figure}


\subsection{K-fold cross-validation results}


\subsection{Regression coefficients}


\subsection{Preditive results}



\end{document}

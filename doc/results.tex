\documentclass[12pt,letterpaper]{article}

\usepackage{amsmath, amsthm, amsfonts, amssymb}
\usepackage{microtype, parskip, graphicx}
\usepackage[comma,numbers,sort&compress]{natbib}
\usepackage{lineno}
\usepackage{longtable}
\usepackage{docmute}
\usepackage{caption, subcaption, multirow, morefloats, rotating}
\usepackage{wrapfig}
\usepackage{hyperref}

\frenchspacing

\begin{document}

\section{Results}

\subsection{Posterior predictive results}
% why do i want to show all these graphs?
%   demonstrate how well the model recapitulates the observed data
%   if the model simulates datasets like the one we observed
%     then we can be more confident in our inferences
%   PPCs by group reveal even more about the fit of our model
%     are some time bins better predicted by others?
%     well predicted time bins indicate congruence with model assumptions
%     poorly predicted time bins indicate incongruence with model assumptions
%       something different is happening in these bins that requires explanation


% overall ppc-s
% mean
\begin{figure}[ht]
  \centering
  \includegraphics[width=\textwidth,height=0.5\textheight,keepaspectratio=true]{figure/ppc_mean}
  \caption{Posterior predictive results comparing the observed mean diversity of a geological unit for each of the studied taxonomic groups to a distribution of 1000 estimates from datasets simulated from the posterior predictive distribution of our models. Model adequacy is determined by how similar the posterior predictive distribution is to the observed value. In all cases, our models appear able to reproduce to observed means.}
  \label{fig:ppc_mean}
\end{figure}

% sd
\begin{figure}[ht]
  \centering
  \includegraphics[width=\textwidth,height=0.5\textheight,keepaspectratio=true]{figure/ppc_sd}
  \caption{Posterior predictive results comparing the observed standard deviation diversity of a geological unit for each of the studied taxonomic groups to a distribution of 1000 estimates from datasets simulated from the posterior predictive distribution of our models. Model adequacy is determined by how similar the posterior predictive distribution is to the observed value. In all cases, our models appear able to reproduce to observed standard deviations.}
  \label{fig:ppc_sd}
\end{figure}

\begin{figure}[ht]
  \centering
  \includegraphics[width=\textwidth,height=0.5\textheight,keepaspectratio=true]{figure/ppc_dens_zoom}
  \caption{Posterior predictive results comparing the empirical probability density of a geological unit for each of the studied taxonomic groups to a distribution of 1000 probability densities from datasets simulated from the posterior predictive distribution of our models. Model adequacy is determined by how similar the posterior predictive distribution is to the observed value. In all cases, our models appear able to almost reproduce to observed ecdf-s.}
  \label{fig:ppc_dens}
\end{figure}

\begin{figure}[ht]
  \centering
  \includegraphics[width=\textwidth,height=0.5\textheight,keepaspectratio=true]{figure/ppc_ecdf}
  \caption{Posterior predictive results comparing the empirical cummulative distribution function (ecdf) of a geological unit for each of the studied taxonomic groups to a distribution of 1000 estimates from datasets simulated from the posterior predictive distribution of our models. Model adequacy is determined by how similar the posterior predictive distribution is to the observed value. In all cases, our models appear able to almost reproduce to observed ecdf-s.}
  \label{fig:ppc_ecdf}
\end{figure}






\subsection{Estimated versus observed unit diversity}
% what is the point of this section?
%   graph unit diversity by time bin
%   compare to estimate from model
%   this is like the ppc for group mean, but inside-out
%   if model has poor estimates for time bin
%     that bin is not like what we would expect based on our model
%   if we have good estimates for all bins
%     then we need to look at covariates to see if there are any switch patterns


% time series graph
% what does this graph represent?
%   geological unit diversity counts at time bins
%   comparison to posterior predictive mean est with 80CI
\begin{figure}[ht]
  \centering
  \includegraphics[width=\textwidth,height=0.5\textheight,keepaspectratio=true]{figure/unitdiv_time}
  \caption{Geological unit diversity though time and the expected diversity (with 80\% credible interval) as estimated from our models. Unit diversity is presented as partially transparent points and our jittered in the y-axis to improve readability. Point size is proportional to the number of units in that interval that have identical unit diversity. The dashed grey line corresponds to the onset of the Hirnantian geological stage, while the dashed-dotted grey line corresponds to the end of the Ordovician epoch and the start of the Silurian epoch.}
  \label{fig:time_div}
\end{figure}

\begin{figure}[ht]
  \centering
  \includegraphics[width=\textwidth,height=0.5\textheight,keepaspectratio=true]{figure/unitdiv_diff}
  \caption{Probability that our estimate of mean unit diversity at time \(t\) is greater than the estimate at time \(t + 1\). The dashed grey horizontal lines correspond to probability of 0.8 and 0.2; these are the thresholds we chose as indicating if a pair-wise difference is potentially larger (or smaller) than no-difference (\(P = 0.5\)), and worthy of further inspection.}
  \label{fig:diff_div}
\end{figure}

% covariance of effect change?



\subsection{Effects of geological covariates on estimated diversity}
% what is the point of this section?
%   we've estimated the effect of multiple covariates
%   these estimates are allowed to vary through time
%     temporal structure is taken into account
%   what is the general relationship? what happens during hirnantian?
%     if all units have the same relationship
%       and all good bin estimates
%         there is no change in unit diversity though time
%         there is no change in geol properties that affect unit diversity
%       and some bad bin estimates
%         our model can not explain these bins; something is going on here
%     if some units have diff relationships (sign change, effect loss)
%       and all good bin estimates
%         we capture the changing relationship between preservational context and observed diversity
%       and some bad bin estimates
%         our model can not explain these bins; something is going on here


% time series graph
% what does this graph represent?
%   effect of covariate on expected count over time
%   violin shows full posterior estimate
%   pointrange shows mean with 80CI
\begin{figure}[ht]
  \centering
  \includegraphics[width=\textwidth,height=0.5\textheight,keepaspectratio=true]{figure/cov_time}
  \caption{Estimates of all estimated covariate effect time series for each of the analyzed taxonomic groups, including intercept estimates. Points represent mean estimate along with a 80\% credible interval. The black horizontal line corresponds to no effect. Points are plotted at the mid-point of the discrete time interval.}
  \label{fig:time_cov}
\end{figure}

\begin{figure}[ht]
  \centering
  \includegraphics[width=\textwidth,height=0.5\textheight,keepaspectratio=true]{figure/cov_diff}
  \caption{Probability that a parameter estimate at time \(t\) is greater than the estimate at time \(t + 1\). The dashed grey horizontal lines correspond to probability of 0.8 and 0.2; these are the thresholds we chose as indicating if a pair-wise difference is potentially larger (or smaller) than no-difference (\(P = 0.5\)), and worthy of further inspection.}
  \label{fig:diff_cov}
\end{figure}

% covariance of effect change?
%   are effects correlated in their changes through time?







\begin{figure}[ht]
  \centering
  \includegraphics[width=\textwidth,height=0.5\textheight,keepaspectratio=true]{figure/compare_pval}
  \caption{Scatterplot of the estimated probability that geological unit diversity is lower during the Hirnantian than either the Ordovician (left facet) or the Silurian (right facet) vs the estimated probability that a covariate estimate is lower during the Hirnantian than either the Ordovician or the Silurian. For each of the taxonomic groups there is only one estimate for the probability of difference in diversity, but there are six probability estimates for each of the covariate effect parameters. }
  \label{fig:<+label+>}
\end{figure}<++>


\end{document}
